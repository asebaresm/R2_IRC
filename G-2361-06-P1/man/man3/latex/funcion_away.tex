Realiza toda la funcionalidad correspondiente con el comando away.


\begin{DoxyCode}
\textcolor{preprocessor}{#include <redes2/irc.h>}
\end{DoxyCode}


Esta funcion se encarga de dejar ausente a un usuario.


\begin{DoxyParams}[1]{Parameters}
\mbox{\tt in}  & {\em id} & identificador del usuario. \\
\hline
\mbox{\tt in}  & {\em usuario} & nombre del usuario. \\
\hline
\mbox{\tt in}  & {\em nick\+\_\+name} & nick actual del usuario. \\
\hline
\mbox{\tt in}  & {\em real} & realname del usuario. \\
\hline
\mbox{\tt in}  & {\em away} & away del usuario. \\
\hline
\mbox{\tt in}  & {\em msg} & mensaje que se quiere mostrar. \\
\hline
\mbox{\tt in}  & {\em I\+Dsocket} & el numero del identificador del socket.\\
\hline
\end{DoxyParams}
\begin{DoxyReturn}{Returns}
No devuelve nada
\end{DoxyReturn}
\begin{DoxyAuthor}{Author}
Celia Mateos de Miguel (\href{mailto:cel.mateos@estudiante.uam.es}{\tt cel.\+mateos@estudiante.\+uam.\+es}) Beatriz de Pablo Garcia (\href{mailto:beatriz.depablo@estudiante.uam.es}{\tt beatriz.\+depablo@estudiante.\+uam.\+es}) Alfonso Sebares Mecha (\href{mailto:alfonso.sebares@estudiante.uam.es}{\tt alfonso.\+sebares@estudiante.\+uam.\+es})
\end{DoxyAuthor}
\begin{DoxyDate}{Date}
13 de febrero de 2017
\end{DoxyDate}


 